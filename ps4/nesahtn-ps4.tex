\documentclass{article}
\usepackage[utf8]{inputenc}
\usepackage{fancyhdr} % Required for custom headers
%\usepackage{lastpage} % Required to determine the last page for the footer
\usepackage{extramarks} % Required for headers and footers
\usepackage[usenames,dvipsnames]{color} % Required for custom colors
\usepackage{graphicx} % Required to insert images
\usepackage{listings} % Required for insertion of code
\usepackage{courier} % Required for the courier font
\usepackage{lipsum} % Used for inserting dummy 'Lorem ipsum' text into the template
\usepackage{enumerate}
\usepackage{multicol}
\usepackage{caption}
\usepackage{subcaption}
\usepackage{ulem} % underline emph
\usepackage{amsmath} % for \text in mathmode
\usepackage[hypcap]{caption}

% Margins
\topmargin=-0.45in
\evensidemargin=0in
\oddsidemargin=0.5in
\textwidth=5.5in
\textheight=9.0in
\headsep=0.25in

\linespread{1.3} % Line spacing

% Set up the header and footer
\pagestyle{fancy}
\lhead{} % Top left header
\chead{\hmwkClass: \hmwkTitle} % Top center head
\rhead{\firstxmark} % Top right header
\lfoot{\lastxmark} % Bottom left footer
\cfoot{\thepage} % Bottom center footer
%\rfoot{Page\ \thepage\ of\ \protect\pageref{LastPage}} % Bottom right footer
\renewcommand\headrulewidth{0.4pt} % Size of the header rule
\renewcommand\footrulewidth{0.4pt} % Size of the footer rule

\setlength\parindent{0pt} % Removes all indentation from paragraphs

\definecolor{MyDarkGreen}{rgb}{0.0,0.4,0.0} % This is the color used for comments
\lstloadlanguages{Matlab} % Load C syntax for listings, for a list of other languages supported see: ftp://ftp.tex.ac.uk/tex-archive/macros/latex/contrib/listings/listings.pdf
\lstset{language=C, % Use python in this example
        frame=single, % Single frame around code
        basicstyle=\small\ttfamily, % Use small true type font
        keywordstyle=[1]\color{Blue}\bf, % C functions bold and blue
        keywordstyle=[2]\color{Purple}, % C function arguments purple
        keywordstyle=[3]\color{Blue}, % Custom functions \underbar underlined and blue
        identifierstyle=, % Nothing special about identifiers
        commentstyle=\usefont{T1}{pcr}{m}{sl}\color{MyDarkGreen}\small, % Comments small dark green courier font
        stringstyle=\color{Purple}, % Strings are purple
        showstringspaces=false, % Don't put marks in string spaces
        tabsize=5, % 5 spaces per tab
        %
        % Put standard Python functions not included in the default language here
        morekeywords={rand},
        %
        % Put Python function parameters here
        morekeywords=[2]{on, off, interp},
        %
        % Put user defined functions here
        morekeywords=[3]{glutCreateWindow,p},
       	%
        morecomment=[l][\color{Blue}]{...}, % Line continuation (...) like blue comment
        numbers=none, % can use none % Line numbers on left
        firstnumber=1, % Line numbers start with line 1
        numberstyle=\tiny\color{Blue}, % Line numbers are blue and small
        stepnumber=1 % Line numbers go in steps of 5
}
% \usepackage{graphicx}
\newcommand{\indep}{\rotatebox[origin=c]{90}{$\models$}}

% Creates a new command to include a perl script, the first parameter is the filename of the script (without .pl), the second parameter is the caption
\newcommand{\code}[1]{
\begin{itemize}
\item[]\lstinputlisting[label=#1]{#1.c}
%\item[]\lstinputlisting[caption=#2,label=#1]{#1.c}
\end{itemize}
}

%----------------------------------------------------------------------------------------
%	DOCUMENT STRUCTURE COMMANDS
%	Skip this unless you know what you're doing
%----------------------------------------------------------------------------------------

\setcounter{secnumdepth}{0} % Removes default section numbers

\newcommand{\homeworkProblemName}{}
\newenvironment{homeworkProblem}[1]{ % Makes a new environment called homeworkProblem which takes 1 argument (custom name) but the default is "Problem #"
    \renewcommand{\homeworkProblemName}{#1} % Assign \homeworkProblemName the name of the problem
    \section{\homeworkProblemName} % Make a section in the document with the custom problem count
}

\newcommand{\problemAnswer}[1]{ % Defines the problem answer command with the content as the only argument
    \noindent\framebox[\columnwidth][c]{\begin{minipage}{0.98\columnwidth}#1\end{minipage}} % Makes the box around the problem answer and puts the content inside
}

\newcommand{\homeworkSectionName}{}
\newenvironment{homeworkSection}[1]{ % New environment for sections within homework problems, takes 1 argument - the name of the section
    \renewcommand{\homeworkSectionName}{#1} % Assign \homeworkSectionName to the name of the section from the environment argument
    \subsection{\homeworkSectionName} % Make a subsection with the custom name of the subsection
}

%----------------------------------------------------------------------------------------
%	NAME AND CLASS SECTION
%----------------------------------------------------------------------------------------

\newcommand{\hmwkTitle}{Problem Set 4} % Assignment title
\newcommand{\hmwkDueDate}{\date{March 14, 2017}} % Due date
\newcommand{\hmwkClass}{TDT4205} % Course/class
\newcommand{\hmwkAuthorName}{Neshat\ Naderi}  % Your name


%----------------------------------------------------------------------------------------
%	TITLE PAGE
%----------------------------------------------------------------------------------------

\title{
\vspace{2in}
\textmd{\textbf{\hmwkClass:\ \hmwkTitle}}\\
\normalsize\vspace{0.1in}\normalsize{\hmwkDueDate}
\vspace{0.1in}\large{\text{Compiler Construction}}
\vspace{3in}
}

\author{\textbf{\hmwkAuthorName}}
\date{} % Insert date here if you want it to appear below your name

%----------------------------------------------------------------------------------------
\begin{document}
\maketitle

% \setcounter{tocdepth}{1} % Uncomment this line if you don't want subsections listed in the ToC

% \newpage
% \tableofcontents
%\newpage

%----------------------------------------------------------------------------------------
%	PROBLEM 1
%----------------------------------------------------------------------------------------

% To have just one problem per page, simply put a \clearpage after each problem
\clearpage

\begin{homeworkProblem}{}
\begin{homeworkSection}{Theory}
\subsubsection{1.1}
TAC representation of the VSL program.\\
\begin{lstlisting}
main:
	BeginFunc 16
		t1 = 2;
		PushParam t1;
		t2 = 3;
		PushParam t2;
		t3 = 4;
		PushParam t3;
		t4 = Lcall axpy;
		PopParam 4;
	EndFunc

axpy:
	BeginFunc 4;
		r1 = a * x;
		r2 = t1 + y;
		Return r2;
	EndFunc		
\end{lstlisting}
\subsubsection{1.2}
On top we have function parameters \texttt{a}, \texttt{x} and \texttt{y}. Below there is storage where local variables could be stored in \texttt{axpy} there are 3 arguments giving to function when called from \texttt{main}. When \texttt{axpy} is called, it will be created 3 local variables which are pushed into the stack. The suggested fram is shown in Table 1.\\
\begin{table}[h!]
\centering
\begin{tabular}{|l|}
\hline
Parameter N                     \\
Parameter N-1                   \\
...                             \\
Parameter 1                     \\ \hline 
Storage for locals \\
and temporaries \\ \hline
\end{tabular}
\caption{Suggested stack frame for a \texttt{func(a,...,n)} function. N represents bytes. }
\label{tab:stack}
\end{table}

\subsubsection{1.3}
\begin{table}[]
\centering
\begin{tabular}{|l|l}
\cline{1-1} 
% Storage on stack &  \\ \cline{1-1} 
Parameter a                     &                \\
Parameter x                     &                \\
Parameter y                     &                \\\cline{1-1} 
Local variable t1               & PushParam t1   \\
Local variable t2               & PushParam t2   \\
Local variable t3               & PushParam t3   \\ \cline{1-1} 
Return address to main function & BeginFunc axpy \\ \cline{1-1} 
\end{tabular}
\caption{Suggested stack frame for \texttt{axpy} function }
\label{my-label}
\end{table}

Table 3 shows the statement of stack about to return. In that statement the return value is stored in EAX. So that stack pointer ESP points to the value stored in local variable r2.
\begin{table}[]
\centering
\label{my-label}
\begin{tabular}{r|l|}
\cline{2-2}
 & $r2 = 10$          \\ \cline{2-2} 

 % EBP $\rightarrow$ &          \\ \cline{2-2} 
ESP $\rightarrow$  & $<$return adr$>$ \\ \cline{2-2} 
 & $t1 = 2$          \\ \cline{2-2} 
 & $t2 = 3$          \\ \cline{2-2} 
 & $t3 = 4$          \\ \cline{2-2} 


\end{tabular}
\caption{Statement of \texttt{axpy} function about to return. }
\end{table}

\end{homeworkSection}

\end{homeworkProblem}


\end{document}

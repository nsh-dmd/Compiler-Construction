\documentclass{article}
\usepackage[utf8]{inputenc}
\usepackage{fancyhdr} % Required for custom headers
%\usepackage{lastpage} % Required to determine the last page for the footer
\usepackage{extramarks} % Required for headers and footers
\usepackage[usenames,dvipsnames]{color} % Required for custom colors
\usepackage{graphicx} % Required to insert images
\usepackage{listings} % Required for insertion of code
\usepackage{courier} % Required for the courier font
\usepackage{lipsum} % Used for inserting dummy 'Lorem ipsum' text into the template
\usepackage{enumerate}
\usepackage{multicol}
\usepackage{caption}
\usepackage{subcaption}
\usepackage{ulem} % underline emph
\usepackage{amsmath} % for \text in mathmode
\usepackage[hypcap]{caption}
\usepackage{tikz}
% Margins
\topmargin=-0.45in
\evensidemargin=0in
\oddsidemargin=0.5in
\textwidth=5.5in
\textheight=9.0in
\headsep=0.25in

\linespread{1.3} % Line spacing

% Set up the header and footer
\pagestyle{fancy}
\lhead{} % Top left header
\chead{\hmwkClass: \hmwkTitle} % Top center head
\rhead{\firstxmark} % Top right header
\lfoot{\lastxmark} % Bottom left footer
\cfoot{\thepage} % Bottom center footer
%\rfoot{Page\ \thepage\ of\ \protect\pageref{LastPage}} % Bottom right footer
\renewcommand\headrulewidth{0.4pt} % Size of the header rule
\renewcommand\footrulewidth{0.4pt} % Size of the footer rule

\setlength\parindent{0pt} % Removes all indentation from paragraphs

\definecolor{MyDarkGreen}{rgb}{0.0,0.4,0.0} % This is the color used for comments
\lstloadlanguages{Matlab} % Load C syntax for listings, for a list of other languages supported see: ftp://ftp.tex.ac.uk/tex-archive/macros/latex/contrib/listings/listings.pdf
\lstset{language=C, % Use python in this example
        frame=single, % Single frame around code
        basicstyle=\small\ttfamily, % Use small true type font
        keywordstyle=[1]\color{Blue}\bf, % C functions bold and blue
        keywordstyle=[2]\color{Purple}, % C function arguments purple
        keywordstyle=[3]\color{Blue}, % Custom functions \underbar underlined and blue
        identifierstyle=, % Nothing special about identifiers
        commentstyle=\usefont{T1}{pcr}{m}{sl}\color{MyDarkGreen}\small, % Comments small dark green courier font
        stringstyle=\color{Purple}, % Strings are purple
        showstringspaces=false, % Don't put marks in string spaces
        tabsize=5, % 5 spaces per tab
        %
        % Put standard Python functions not included in the default language here
        morekeywords={rand},
        %
        % Put Python function parameters here
        morekeywords=[2]{on, off, interp},
        %
        % Put user defined functions here
        morekeywords=[3]{glutCreateWindow,p},
       	%
        morecomment=[l][\color{Blue}]{...}, % Line continuation (...) like blue comment
        numbers=none, % can use none % Line numbers on left
        firstnumber=1, % Line numbers start with line 1
        numberstyle=\tiny\color{Blue}, % Line numbers are blue and small
        stepnumber=1 % Line numbers go in steps of 5
}
% \usepackage{graphicx}
\newcommand{\indep}{\rotatebox[origin=c]{90}{$\models$}}

% Creates a new command to include a perl script, the first parameter is the filename of the script (without .pl), the second parameter is the caption
\newcommand{\code}[1]{
\begin{itemize}
\item[]\lstinputlisting[label=#1]{#1.c}
%\item[]\lstinputlisting[caption=#2,label=#1]{#1.c}
\end{itemize}
}

%----------------------------------------------------------------------------------------
%	DOCUMENT STRUCTURE COMMANDS
%	Skip this unless you know what you're doing
%----------------------------------------------------------------------------------------

\setcounter{secnumdepth}{0} % Removes default section numbers

\newcommand{\homeworkProblemName}{}
\newenvironment{homeworkProblem}[1]{ % Makes a new environment called homeworkProblem which takes 1 argument (custom name) but the default is "Problem #"
    \renewcommand{\homeworkProblemName}{#1} % Assign \homeworkProblemName the name of the problem
    \section{\homeworkProblemName} % Make a section in the document with the custom problem count
}

\newcommand{\problemAnswer}[1]{ % Defines the problem answer command with the content as the only argument
    \noindent\framebox[\columnwidth][c]{\begin{minipage}{0.98\columnwidth}#1\end{minipage}} % Makes the box around the problem answer and puts the content inside
}

\newcommand{\homeworkSectionName}{}
\newenvironment{homeworkSection}[1]{ % New environment for sections within homework problems, takes 1 argument - the name of the section
    \renewcommand{\homeworkSectionName}{#1} % Assign \homeworkSectionName to the name of the section from the environment argument
    \subsection{\homeworkSectionName} % Make a subsection with the custom name of the subsection
}

%----------------------------------------------------------------------------------------
%	NAME AND CLASS SECTION
%----------------------------------------------------------------------------------------

\newcommand{\hmwkTitle}{Problem Set 6} % Assignment title
\newcommand{\hmwkDueDate}{\date{April 27, 2017}} % Due date
\newcommand{\hmwkClass}{TDT4205} % Course/class
\newcommand{\hmwkAuthorName}{Neshat\ Naderi}  % Your name


%----------------------------------------------------------------------------------------
%	TITLE PAGE
%----------------------------------------------------------------------------------------

\title{
\vspace{2in}
\textmd{\textbf{\hmwkClass:\ \hmwkTitle}}\\
\normalsize\vspace{0.1in}\normalsize{\hmwkDueDate}
\vspace{0.1in}\large{\text{Compiler Construction}}
\vspace{3in}
}

\author{\textbf{\hmwkAuthorName}}
\date{} % Insert date here if you want it to appear below your name

%----------------------------------------------------------------------------------------
\begin{document}
\maketitle

%  \setcounter{tocdepth}{1} % Uncomment this line if you don't want subsections listed in the ToC

% \newpage
% \tableofcontents
%\newpage

%----------------------------------------------------------------------------------------
%	PROBLEM 1
%----------------------------------------------------------------------------------------

% To have just one problem per page, simply put a \clearpage after each problem
\clearpage

\begin{homeworkProblem}{}
\section{Theory}

\subsection{1.}
\begin{enumerate}[(a)\ \ ]
\item \texttt{for(a; b; c) d; e;}
\item \texttt{a; while(b)\{d; c;\} e;}
    \item \texttt{a; do\{d ; c; \} while(b); e;}
\end{enumerate}
    \begin{figure}[h]
    \centering
            \begin{subfigure}{0.4\textwidth}
        \includegraphics[width=0.5\linewidth, height=4cm]{cfg1}
        \caption{}
        \label{fig:cfg1}
        \end{subfigure}
        \begin{subfigure}{0.45\textwidth}
        \includegraphics[width=0.5\linewidth, height=4cm]{cfg2}
        \caption{}
        \label{fig:cfg2}
        \end{subfigure}
        \begin{subfigure}{0.35\textwidth}
        \includegraphics[width=0.5\linewidth, height=5cm]{cfg3}
        \caption{}
        \label{fig:cfg3}
        \end{subfigure}
    \caption{Problem 1.1. Control flow graphs.}
    \label{fig:mult}
    \end{figure}
\clearpage
\subsection{2.}
Program fragment : \\
\begin{verbatim}
for ( i=0; i<n; i++ ) {                 
    int sum =4*i; 
    for (int j=0; j<m; j=j+i ) {
        a = a + b * 2;
    }
}
\end{verbatim}
\subsubsection{2.1 Control flow graph with three-address instructions}
\begin{verbatim}
node 1:     i = 0
node 2:     if i < n     
node 3:     sum = 4 * i
node 4:     j = 0
node 5:     if j < m
node 6:     t1 = b * 2
node 7:     a = a + t1
node 8:     j = j + 1
node 9:     i = i + 1
node 10:    exit
\end{verbatim}

\begin{table}[h!]
\begin{tabular}{c|c|c}
Label(\textit{e} &  Node & Expression \\ \hline \hline
    $e_1$ &  \tikz \node[circle, draw]{3}; & \texttt{4 * i} \\
    $e_2$ &  \tikz \node[circle, draw]{6}; & \texttt{b * 2} \\
    $e_3$ &  \tikz \node[circle, draw]{7}; & \texttt{a + t1} \\
    $e_4$ &  \tikz \node[circle, draw]{8}; & \texttt{j + i} \\
    $e_5$ &  \tikz \node[circle, draw]{9}; & \texttt{i + 1} \\
\end{tabular}
\end{table}

\begin{figure}[h!]
\centering
    \includegraphics[width=.5\linewidth]{tac}
\caption{Problem 1.2. Control flow graph with three-adress encoding.}
\label{fig:tac}
\end{figure}

\clearpage

\subsubsection{2.2 Dataflow equations for the dominator relation}


$
D(1) = \{1\}   \\
D(2) = \{2\} \cup (D(1) \cap D(9)) = \{2\}\cup (\{1\} \cup \{1,2,3,4,5,9\}) = \{1,2\}  \\
D(3) = \{3\} \cup D(2) = \{1, 2, 3\} \\  
D(4) = \{4\} \cup D(3) = \{1,2,3,4\} \\
D(5) = \{5\} \cup (D(4)\cap D(8)) = \{5\} \cup (\{1,2,3,4\} \cap \{1,2,3,4, 5, 6, 7, 8\} = \{1,2,3,4, 5\} \\
D(6) = \{6\} \cup D(5) = \{1,2,3,4, 5, 6\}\\
D(7) = \{7\} \cup D(6) = \{1,2,3,4, 5, 6, 7\}\\
D(8) = \{8\} \cup D(7) = \{1,2,3,4, 5, 6, 7, 8\} \\
D(1) = \{1\} \cup D(8) = \{1,2,3,4, 5, 6, 7, 8, 9\}\\
D(10) = \{10\} \cup (D(2) \cap D(1)) = \{1, 2, 10\} 
$ 
    


\subsubsection{2.3 Dominator tree}
\begin{figure}[h!]
\centering
    \includegraphics[width=0.5\linewidth]{tree}
\caption{Problem 1.2. Dominator tree.}
\label{fig:tac}
\end{figure}

\end{homeworkProblem}


\end{document}


